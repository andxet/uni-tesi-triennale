% !TEX encoding = UTF-8 Unicode
\chapter{Strumenti utilizzati}
Nella realizzazione del progetto sono stati affrontati molti problemi, in alcuni casi adottando soluzioni comuni nel mondo della programmazione online\footnote{Ad esempio nella manipolazione del DOM, è stata utilizzata una libreria JavaScript chiamata JQuery}, altre realizzate in modo specifico per il progetto. Le problematiche affrontate, tenendo conto delle richieste dei fornitori del dataset, sono state quelle di mostrare i dati nel modo più esplicativo possibile senza fornire la possibilità di accedere direttamente al dataset, utilizzando solo l'interfaccia front-end sviluppata. Il programma quindi effettua delle richieste asincrone al server per ottenere alcune parti del dataset, formattate in modo da contenere solo i dati necessari alla visualizzazione corretta sulla mappa dei punti di interesse.

Un altri obbiettivi da raggiungere sono stati:
\begin{itemize}
	\item Rendere possibile una crescita del progetto in futuro

Il progetto è stato strutturato in modo da rendere possibile in futuro l'aggiunta di funzioni ed aggiungere nuovi locali, tenendo ben presente che i pochi record attuali sono molto semplici da gestire, mentre se dovessero aumentare di molto il programma è comunque capace di gestire i nuovi record senza appesantirsi di un overhead troppo elevato.

	\item Utilizzare un database

Siccome il dataset iniziale era un semplice file CSV che necessitava di molte correzioni, l'utilizzo di un database non relazionale come Mongo DB (Capitolo \ref{sec:mongo} a pag. \pageref{sec:mongo} ) avrebbe semplificato di molto la gestione dei record, permettendo di effettuare con la programmazione una più semplice pulizia dei dati. La natura del database permette di memorizzare i record anche se alcuni non contengono informazione in tutti i campi, oppure se alcuni dei campi non sono presenti.

\end{itemize}


	\section{Software lato server}%decidere se presentare prima il pattern o lo strumento
		\subsection{Turbogears}
		\subsection{Mongo DB}\label{sec:mongo}
		\subsection{Jinja}%Controllare come si chiama il template engine
	\section{Linguaggi web}
	\section{Librerie}
		\subsection{d3.js}\label{sec:d3}
		\subsection{polymaps}\label{sec:polymaps}
		\subsection{jquery}
	\section{Altri strumenti}
		%Sito da cui otteniamo le mappe
			
	\section{Funzionamento}%richiesta al server tramite url, AJAX per la richiesta dei dati, come opera turbogears, problemi e soluzioni
