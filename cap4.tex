% !TEX encoding = UTF-8 Unicode
\chapter{Architettura del sistema e sue funzionalità}%richiesta al server tramite url, AJAX per la richiesta dei dati, come opera turbogears, problemi e soluzioni
Le problematiche affrontate, tenendo conto delle richieste dei fornitori del dataset, sono state quelle di mostrare i dati nel modo più esplicativo possibile senza fornire la possibilità di accedere direttamente al dataset, utilizzando solo l'interfaccia front-end sviluppata. Il programma carica una sola volta la pagina web ed effettua delle richieste asincrone al server per per interrogare il database, ottenendo risultati tali da contenere solo i dati necessari alla visualizzazione corretta sulla mappa dei punti di interesse.

Altri obbiettivi da raggiungere sono stati:
\begin{itemize}
	\item Rendere possibile una crescita del progetto in futuro

Il progetto è stato strutturato in modo da rendere possibile in futuro l'aggiunta di funzioni ed aggiungere nuovi locali, tenendo ben presente che i pochi record attuali sono molto semplici da gestire, mentre se dovessero aumentare di molto il programma è comunque capace di gestire i nuovi record senza appesantirsi troppo. L'utilizzo di framework web, oltre a velocizzare enormemente lo sviluppo, offre di strutturare il progetto modularmente e permettendo così di estenderlo in futuro se sono necessarie altre features.

	\item Utilizzare un database

Siccome il dataset iniziale era un semplice file CSV che necessitava di molte correzioni, l'utilizzo di un database non relazionale come Mongo DB (Capitolo \ref{sec:mongo} a pag. \pageref{sec:mongo} ) avrebbe semplificato di molto la gestione dei record, permettendo di effettuare con la programmazione una più semplice pulizia dei dati. La natura del database permette di memorizzare i record anche se alcuni non contengono informazione in tutti i campi, oppure se alcuni dei campi non sono presenti. Tramite l'utilizzo di funzioni scritte appositamente per interrogare il DB ed ottenere parti precise di dati, si risolve il problema della non distribuzione del dataset.
La struttura del progetto è quella standard di Turbogears ed è stata discussa nel paragrafo \ref{sec:tg} ed in particolare si faccia riferimento all'immagine \ref{fig:progetto}.

\end{itemize}

Nel progetto i linguaggi di programmazione usati prevalentemente sono stati Python per la parte lato server e Javascript lato client. Non mancano ovviamente i linguaggi web come HTML e CSS.

\section{Lato client}
Una volta scaricata la pagina, al posto della mappa sarà presente un tag \texttt{div} con id="map" vuoto, che sarà completato dalla chiamata a funzione \texttt{drawMap()} a fondo pagina\footnote{Una buona tecnica per evitare che il caricamento della pagina rimanga bloccato a causa del download degli script, è quello di spostare i tag \texttt{script} alla fine del documento.} che si occuperà di creare le strutture dati necessarie a mostrarla. 
\texttt{drawMap} è una funzione javascript definita nel file \texttt{public/js/map.js} che esegue queste operazioni:
\begin{enumerate}
\item seleziona nel DOM il tag \texttt{div} prima citato
\item crea l'oggetto della mappa, settando la posizione nella pagina usando la variabile del punto 1, le coordinate geografiche a cui puntare e lo zoom.
\item assegna all'oggetto mappa la sorgente delle immagini raffiguranti la mappa vera e propria.
\item aggiungono i pulsanti per regolare lo zoom, per navigare nei risultati, ed il riquadro che visualizzerà i dettagli dei locali.
\item aggiunge l'oggetto mappa all'interno del tag \texttt{div} con id="map", visualizzandolo effettivamente sulla mappa.
\end{enumerate}

I locali però non sono ancora stati caricati a questo punto, ma nemmeno le città ed i tipi selezionabili. La funzione \texttt{json()} di D3 ha la capacità di reperire dati in formato JSON dall'url passato come parametro. Il JSON, acronimo di JavaScript Object Notation, è un formato adatto per lo scambio dei dati in applicazioni client-server\footnote{cit. Wikipedia \url{http://it.wikipedia.org/wiki/JSON}} e consiste nel linguaggio utilizzato per definire oggetti in javascript. Tramite due chiamate alla funzione \texttt{json()} si reperiscono dal server l'elenco delle città e dei tipi, successivamente si inserisce ogni opzione nelle liste sopra la mappa.

Infine, si devono visualizzare i segnalini dei locali sulla mappa. La funzione \texttt{updateViz()} si occupa di controllare quali opzioni sono selezionate nei menu a tendina delle città e dei tipi, ed effettuare una richiesta al server, tramite la funzione \texttt{json()} di D3. I risultati ottenuti rappresentano alla prima chiamata tutti i locali presenti nel database, ed è necessario creare un puntino blu per ognuno di essi sulla mappa.

D3 permette di definire funzioni che operano su liste, ed è in grado, con la creazione una nuova lista, di mappare gli elementi già presenti con quella vecchi, permettendo quindi di rilevare gli elementi nuovi e quelli non più presenti. Come mostrato nel listato \ref{ls:enter} D3 dispone di funzioni chiamate \texttt{enter} ed \texttt{exit} che rispettivamente definiscono il codice da eseguire per gli elementi nuovi e quelli non più presenti. Per ogni elemento nuovo si visualizzerà un pallino creando un tag \texttt{circle}, dettando dimensione, colore e cosa deve accadere nel caso in cui si faccia click su di esso. Le operazioni da applicare agli elementi uscenti dicono al programma di avviare un'animazione che porta la trasparenza a zero, facendoli svanire, per poi eliminare i tag \texttt{circle} dal DOM. Questo sistema di \emph{enter} ed \texttt{exit} è un'alternativa a scrivere un ciclo e permette di scrivere direttamente cosa deve fare D3 con ogni elemento.
\begin{lstlisting}[label=ls:enter,caption={\texttt{enter()} ed \texttt{exit()} di d3.js},language=c]
d3.json('points.json?city=' + settings.city + '&type=' + settings.type, function(points){
    selection.setData(points['data'])
    showDefaultDetails()
    // Add an svg:g for each station.
    var marker = layer.selectAll(".gcircle")
        .data(dat, function(d){
            return d._id
            })
        
    marker.enter()
        .append("g")
        .on("click", selection.selectPoint)
        .attr('class', 'gcircle')
        .append("circle")
            .attr('class','circle')
            .attr("r", 3);
    
    marker.exit()
        .transition()
        .style('opacity', 0)        
        .duration(1000)
        .remove()
\end{lstlisting}

Viene poi mossa la mappa in modo che il punto \( (x, y) \) sia al centro, ricavato dalla funzione \ref{eq:centro} che calcola il punto medio tra i valori maggiore e minore di latitudine e longitudine. Infine si regola il focus in modo che tutti i punti siano visibili.
\begin{equation}
\label{eq:centro}
(x, y)=\left(\frac{max(lat)+min(lat)}{2}, \frac{max(lon)+min(lon)}{2}\right)
\end{equation}

A questo punto il programma aspetta input da parte dell'utente, e la mappa è interattiva. Grazie all'evento \texttt{move} polymaps rivela i drag del mouse sulla mappa, con il risultato di spostare la mappa ed i marker nella direzione dello spostamento del mouse, i tasti \texttt{+} e \texttt{-} in alto a sinistra permettono di modificare lo zoom, e cliccando su un marker verrà selezionato.
Quando un locale rappresentato da un punto viene selezionato, viene richiamata la funzione \texttt{selection} e mostrata un'animazione in cui la vecchia selezione torna un normale punto mentre quello nuovo cambia colore e si ingrandisce. D3 invia poi una richiesta al server contenente l'ID del locale e riceve le informazioni riguardanti il \emph{nome}, \emph{tipo}, \emph{periodo storico}, \emph{indirizzo} e \emph{comune}, e li visualizza nel box in alto a destra. 

Tramite un url che permette di ottenere un'immagine tratta da Google Street View che mostra una visuale di un certo indirizzo, viene mostrato il luogo che secondo Google Maps corrisponde all'indirizzo memorizzato. Questa feature è stata inserita sperimentalmente, ma non offre una panoramica soddisfacente del posto e non tutti gli indirizzi hanno un'immagine associata.

\`E possibile selezionare i locali uno per uno, utilizzando i pulsanti di navigazione in alto al centro. La pressione di quei pulsanti non farà altro che aumentare o diminuire l'iterator che punta all'oggetto selezionato nella lista dei risultati e richiamare la funzione \texttt{selection} su di esso. \label{sec:riposizionamento}Il focus della mappa si sposta in modo da puntare il marker selezionato, e farlo apparire al centro della mappa.

L'ultima funzionalità è quella della ricerca per comune e/o tipo. Selezionando un elemento nel menu \texttt{select} dei comuni la lista dei tipi viene modificata mostrando solo i tipi disponibili in quel comune, sempre grazie a d3 ed una funzione scritta nel server. Premendo il tasto \emph{filtra}, verrà richiesto al server l'elenco di locali che soddisfano i criteri selezionati, e si affideranno i risultati a d3, il quale provvederà a visualizzare i nuovi punti ed eliminando quelli non più presenti come descritto in precedenza.

\section{Lato server}
Solitamente quando un server web riceve una richiesta in cui la pagina non è specificata, controlla l'esistenza di un file chiamato \emph{index} e lo seleziona come risposta. Nel caso di Piedmont Heritage non è una pagina, ma una funzione chiamata \texttt{index}, riportata nel listato \ref{ls:index}. Il decoratore specificato indica il file \texttt{index.html} nella cartella \texttt{templates}. La natura ad oggetti del framework permette di considerare un oggetto anche una pagina web, in questo caso è stato utilizzato l'oggetto che lo rappresenta.
\begin{lstlisting}[label=ls:index,caption={root controller, funzione index, richiamata alla richiesta della pagina web da parte del client.},language=Python]
@expose('pheritage.templates.index')
def index(self):
    """Handle the front-page."""
    return dict(page='index')
\end{lstlisting}

Si possono ottenere parametri dal client in diversi modi\footnote{In modo simile a quanto descritto in seguito, è possibile ottenere i parametri inviati tramite i metodi \emph{GET} e \emph{POST}.}, il progetto li inserisce in coda all'url dopo il nome della funzione (vedi paragrafo \ref{sec:tg}). Nel listato \ref{ls:index} non sono presenti parametri, quindi non ci si aspetta valori passati dalla richiesta HTTP. Nella funzione \ref{ls:param} a pagina \pageref{ls:param} ci si aspetta un parametro \texttt{id}, necessario per individuare il locale nel database e richiederne le informazioni.

Ming come già detto si occupa delle operazioni su database. Esso prevede due modalità, una più a basso livello, l'altra più ad alto livello con maggiore astrazione utilizzata dal progetto, chiamata \emph{ODM}. Esso fornisce alcune funzionalità in più dell'altra versione, tra cui un sistema che offre alcune funzionalità simili alle transazioni di SQL, e la comodità di operare sugli oggetti veri e propri nel caso di campi che puntano ad altri documenti (chiavi esterne)\footnote{Maggiori dettagli su \url{http://merciless.sourceforge.net/odm.html}}. Richiede la definizione delle classi che rappresenteranno il model. Esse dovranno estendere la \texttt{MappedClass} di Ming, definire una sottoclasse contenente le informazioni necessarie al framework e definire i campi della classe, che saranno effettivamente i campi che verranno memorizzati nel database (vedi esempio nel listato \ref{ls:mingclass}).

\begin{lstlisting}[label=ls:mingclass,caption={Classe della collezione di categorie, con i campi da salvare sul database.},language=Python]
class Categoria(MappedClass):
    class __mongometa__:
        session = DBSession
        name = 'categorie'
        
    _id         = FieldProperty(s.ObjectId)
    nome     = FieldProperty(s.String)
    cont       = FieldProperty(s.Int)
\end{lstlisting}

La definizione di classi del model permette anche di specificare campi con riferimenti ad altri documenti, anche in altre collezioni, ma in questo progetto non sono stati utilizzati.

Per il salvataggio e la ricerca su database è necessario introdurre il concetto di \emph{sessione} di Ming: consiste nel collegamento al database, ma fornisce anche i metodi per interagire con esso. Ogni operazione sugli oggetti mappati con il database finisce in una coda di operazioni, che verrà scritta effettivamente quando si richiamerà il metodo \texttt{flush()}. Mongo non supporta le transazioni come SQL, questa funzionalità cerca di ricalcare alcuni benefici da loro offerti.

Per creare nuovi documenti occorre istanziare la classe prima definita ed effettuare una chiamata al metodo \texttt{flush()} della sessione. I dati dell'applicazione inizialmente risiedono in un file json\footnote{Frutto del lavoro di pulizia dei colleghi di progetto, nda.} contenente tutti i dati dei Locali, che vengono inseriti nel database durante la procedura di installazione del programma durante il \emph{bootstrap}, una funzione che effettua le operazioni preliminari prima del primo avvio del programma, definita nel file \texttt{bootstrap.py}. Essa analizza il file e carica i locali uno per uno.

L'interrogazione del database avviene con il metodo \texttt{find()} dell'oggetto sessione: accetta due parametri, con il primo si può specificare tramite un oggetto \texttt{dict} di python una query, in modo simile a quelle da effettuare sul client di Mongo in Javascript, permettendo l'equivalente di una \emph{selezione} nel calcolo relazionale. Il secondo permette di specificare quali campi ottenere, se omesso vengono restituiti tutti. L'equivalente in calcolo relazionale di una \emph{proiezione}.

In Piedmont Heritage l'interrogazione del database avviene, prendendo in considerazione il listato \ref{ls:param}, utilizzando l'oggetto \texttt{c\_locale}. Esso è l'equivalente dello scrivere \texttt{model.mainsession.db.locali}, ed è l'oggetto che fornisce il metodo \texttt{find()} per la collezione \emph{locali}


Successivamente verranno descritte le funzioni del root controller ed il loro funzionamento.
(TODO: decidere se lasciarlo) Il listato completo del root controller, presente nel file texttt{controllers/root.py} è a pagina \pageref{ls:completo}.

\begin{lstlisting}[label=ls:param,caption={Funzione utilizzata per richiedere i dettagli su un locale con ID=\texttt{id}},language=Python]
@expose('json')
def detail(self, id):
    det = c_locale.find({'_id': ObjectId(id)})
    return dict(data = list(det))
\end{lstlisting}

La funzione del Listato \ref{ls:param} è utilizzata dalla funzione richiamata dal client quando viene selezionato un marker. Il decoratore specificato (il parametro della direttiva \texttt{@expose()}) è JSON, la funzione restituirà quindi il valore di ritorno creando un oggetto in formato JSON equivalente. L'url a cui il client riceve i dati è la seguente:
\begin{lstlisting}
http://piedmontheritage.com/detail/527ac93d3573d7707e1a2f27.json
\end{lstlisting}
\texttt{detail} è il nome della funzione, mentre \texttt{527ac93d3573d7707e1a2f27} è l'id del locale. La risposta del server, in JSON, è quella del listato \ref{ls:detail}.

\begin{minipage}{\linewidth}
\begin{lstlisting}[label=ls:detail,caption={Dati in formato JSON restituiti dal server.},language=c]
{"data": [
    		{"comune": "ACQUI TERME",
    	 	 "indirizzo": "Via G. Garibaldi 73",
    	 	 "datazione": "1930",
    	 	 "tipo": "Merceria",
    	 	 "images": [],
    	 	 "lon": 8.4677381,
    	 	 "denom": "Merceria Voglino",
    	 	 "lat": 44.6762883,
    	 	 "denomlocale": "",
    	 	 "restauro": "1980",
    	 	 "_id": "527ac93d3573d7707e1a2f27",
    	 	 "id": "R0335027","
    	 	 ": "AL"}
    	    ]
}
\end{lstlisting}
\end{minipage}

(TODO: scegliere se descrivere ogni funzione, oppure fare una descrizione generale.)
Le altre funzioni del controller funzionano in modo analogo:

La funzione \texttt{categories()} si occupa di estrarre dal database l'elenco di tutte delle categorie. Questa funzione è usata quando il client richiede tutte le categorie senza filtrarle per comune.

La funzione \texttt{cities()} effettua una ricerca ricavando l'elenco dei comuni, eliminando le voci duplicate ed ordinadole in redine alfabetico.

La funzione \texttt{types(comune)} si occupa di verificare se il valore di \texttt{comune} è \emph{tutti i comuni}, in tal caso estrae l'elenco di tutti i tipi, mentre se è selezionato un comune estrae solo quelli presenti in quel comune.

La funzione \texttt{points(comune, tipo)} interroga il database ottenendo l'elenco di locali che soddisfano i termini di ricerca definiti da \texttt{comune} e \texttt{tipo}.


\lstinputlisting[label=ls:completo,caption={Codice completo del file \texttt{controllers/root.py}.},language=Python]{ls/root.py}


