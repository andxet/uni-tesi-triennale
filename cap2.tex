% !TEX encoding = UTF-8 Unicode
\chapter{Concetti}%Decidere se trasformare in un capitolo di teoria, in cui si parla anche del pattern MVC
In questo capitolo verranno trattati alcuni concetti teorici la cui conoscenza è necessaria per fornire un quadro generale degli argomenti trattati. Si parlerà del \emph{modello client/server} il quale è alla base del funzionamento di un sito web, in cui un calcolatore chiamato \emph{server} ha il compito di fornire ai calcolatori chiamati \emph{client} i dati richiesti, oppure eseguire delle operazioni per loro. Si parlerà del \emph{pattern MVC}, un metodo di strutturazione del software molto utilizzato. Si parlerà infine dei \emph{BigData} e delle tre parti trattate nel capitolo~ \ref{cap:introduzione}.
\section{Modello Client/server}%Veramente serve??
Veramente serve?

\section{Pattern MVC}
In informatica un pattern è TODO: inserire una definizione tratta da un libro e la sua citazione nella bibliografia
Il pattern MVC si occupa di definire una struttura per il software che divide e rende autonome le parti di visualizzazione, elaborazione e memorizzazione dei dati.
TODO: Inserire una figura che rappresenta il pattern mvc.
Questa struttura offre notevoli vantaggi in termini di pulizia, estensione e mantenimento del codice. Il codice viene diviso in più parti ognuno delle quali sarà un Il programma sarà più pulito siccome in ogni sorgente sarà presente solamente il codice che si occupa di uno specifico compito. Non si presenterà quindi una situazione in cui nella stessa parte di codice oltre ad elaborare i dati ci dobbiamo preoccupare di formattarli per la presentazione su schermo, perché sarà compito della \emph{View}. L'aggiunta do nuove funzioni al software è molto semplice, per effettuare delle modifiche è spesso necessario aggiungere poche righe di codice in ogni parte del programma, senza il bisogno di stravolgerne la struttura. Il mantenimento del software risulta più semplice grazie alla suddivisione del codice in base al ruolo ed alla conseguente leggibilità.
\subsection{Model}
Il model, o \emph{modello} è la parte che si occupa della memorizzazione dei dati dell'applicazione, fornendo delle funzioni per salvare, leggere e modificare i dati dalla base dei dati. Le funzioni fornite sono l'\emph{interfaccia} attraverso cui il resto del programma può interagire con i dati dell'applicazione, non è possibile farlo direttamente sul database oppure in altri modi: rappresenta quindi uno strato al di sopra della base di dati, che ha il compito di fornire metodi ad alto livello per l'accesso ai dati, e di proteggere gli stessi da errori di programmazione comportati dall'accesso diretto.
\subsection{Controller}
Questa sezione si occupa di interfacciarsi con le altre due. Un controller in genere riceve le richieste dell'utente sotto forma di chiamate a funzioni e si occupa di eseguirle, preoccupandosi della validazione e della coerenza dei dati. Esso può essere un tramite tra \emph{View} e \emph{Model}, oppure operare solo con uno di loro:
\begin{itemize}
	\item Aggiunta di nuovi dati o modifica di dati esistenti
	
	Quando l'utente inserisce alcuni dati in un form e desidera che vengano memorizzati sul database, la View incapsula i dati inseriti in opportune strutture dati che verranno passate al controller, il quale effettuerà i controlli necessari per verificare che i dati non contengano errori, assicurarsi che non confliggano con quelli memorizzati, per poi inoltrare una richiesta di memorizzazione al Model, il quale eseguirà in modo del tutto trasparente al Controller l'operazione. Infine, il Controller invierà una conferma o l'errore alla View, in modo di informare l'utente su cosa ha fatto il sistema.
	\item Visualizzazione di dati
	
	In questo caso, la view invia una richiesta contenente le informazioni riguardanti a quali dati l'utente desidera visualizzare al Controller, quest'ultimo si occupa di verificare se l'utente è autorizzato ad accedere a quei dati restituendo errore nel caso in cui non sia autorizzato, oppure, se il Model supporta un sistema di gestione dei permessi ad utenti, inoltra la richiesta al model aggiungendo le informazioni sull'identità dell'utente.Il Model a questo punto provvederà a recuperare i dati e restituirli al Controller, effettuare alcune operazioni su di loro se necessario, per poi inviarle alla View.
\end{itemize}

I Controller sono quindi quelle parti del programma che si trovano tra l'interfaccia utente e la base di dati, che si occupano di coordinare le altre due componenti. Il loro compito è eseguire le richieste degli utenti, utilizzando le interfacce fornite da View e Model per portarle a termine, ma preoccupandosi anche dei controlli sui dati che transitano in essi.
\subsection{View}
La View è la parte che si occupa della \emph{visualizzazione} dei dati forniti dal controller. Essa utilizza nel sorgente dei \emph{placeholder} che verranno sostituiti a runtime dai dati.
Un esempio in campo web, potrebbe essere una semplice pagina che saluta il cliente e visualizza un elenco di prodotti in vendita:
\begin{verbatim}
#View
Ciao {{nome}}!
Oggi in vendita abbiamo: {{for frutto in frutti}}
\end{verbatim}
Le parti tra le doppie parentesi graffe\footnote{La sintassi di questo esempio è ispirata a TODO: Cosa? chiamato Mustache TODO: Link al sito nella bibliografia} sono i placeholder, i quali possono contenere anche semplici istruzioni di ciclo o di confronto. Nel momento dell'esecuzione queste verranno sostituite con i dati indicati. Questo è necessario perché al momento della scrittura del programma le variabili "nome" e "frutti" non sono ancora stati calcolati!
Se i dati forniti dal controller sono questi:
\begin{verbatim}
#Controller
nome="Andrea"
frutti=['pere', 'mele', 'arancie']
\end{verbatim}
Il risultato dell'esecuzione sarà quindi:
\begin{verbatim}
#Risultato dell'esecuzione:
Ciao Andrea!
Oggi in vendita abbiamo: pere mele arancie
\end{verbatim}


\section{Big Data}\label{sec:bigdata}
I Big Data
\begin{quotation}
sono dati che superano i limiti degli strumenti di database tradizionali.\footnote{\cite{rezzani2013big}}
\end{quotation} 
\subsection{Acquisizione}
\subsection{Elaborazione}
\subsection{Visualizzazione}