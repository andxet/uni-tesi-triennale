% !TEX encoding = UTF-8 Unicode
\chapter{Conclusioni}

Le soluzioni presentate sono ovviamente solo alcune delle soluzioni possibili, e sono state scelte in seguito al contesto dello sviluppo del progetto e per sperimentare nuove tecnologie. 
In questo testo abbiamo visto la soluzione adottata per la visualizzazione di dati georeferenziati con attenzione alla possibilità di crescita della quantità dei dati. Dopo un'introduzione all'argomento nel cap. \ref{cap:introduzione}, nel cap. \ref{cap:concetti} si è vista una definizione di BigData ed una descrizione delle tre fasi che caratterizzano l'elaborazione di grandi moli di dati, una descrizione del pattern MVC e gli strumenti utilizzati. Nel cap. \ref{sec:piedmontheritage} è stato introdotto il progetto Piedmont Heritage e descritto insieme al suo contesto, Mentre nel cap. \ref{cap:funzionamento} è stato descritto nei minimi particolari il funzionamento "sotto il cofano" del progetto, mostrando le soluzioni adottate per la risoluzione di numerosi problemi utilizzando gli strumenti presentati nei cap. \ref{sec:server} e \ref{sec:librerie}. Infine, nel cap. \ref{cap:azione} è stata presentata una sessione di esempio corredata di screenshot e descrizione di ciò che succedeva sullo schermo ed del codice che veniva eseguito.